Die folgenden Unterkapitel klären über die Entwicklungsarbeiten auf, welche für das Umsetzen des \acrshort{tof}-Demonstrators
nötig sind. Hierbei wird vor allem darauf eingegangen, wie die Elektronik aufgebaut sein kann. Zudem werden weitere
Herausforderungen eines solchen Systems und die mögliche Bewältigung ebendieser aufgezeigt.

\subsection{Signalkette}

Nachfolgend sind einige Erklärungen dazu zu finden, wie die \acrshort{tof}-Messung realisiert wird. Eine Skizze der Signalkette ist in der
Abbildung~\ref{fig:konzept_signalpfad} ersichtlich.

\begin{figure}[H]
    \centering
    \includegraphics[width=\textwidth]{diagrams/konzept_signalpfad.pdf}
    \caption{Konzeptioneller Signalpfad}\label{fig:konzept_signalpfad}
\end{figure}

Eine \acrshort{mcu} kümmert sich um das Starten und Auswerten der Messresultate. Das Generieren eines Startsignals hat
zur Folge, dass über einen Laser-Treiber eine Laser-Diode mit einem Puls angesteuert wird. Dieser Puls, nach der Laser-Diode
in Form eines Lichtpulses, wird dann von einem Target reflektiert, welches in einer variablen Distanz zur Elektronik steht.

Der reflektierte Lichtpuls wird über eine Optik fokussiert und anschliessend von einer Photo-Diode in einen
Lichtstrom umgewandelt. Dieser Lichtstrom wird von einem Transimpedanzverstärker in eine Spannung umgewandelt. Bevor die
\acrshort{mcu} diese Spannung auswerten kann, muss diese erst digitalisiert werden. Dies passiert unter Verwendung eines Komparators.
Dieser kann als 1~bit \acrshort{adc} verstanden werden, da bloss eine Unterscheidung der beiden Zustände \dq Licht\dq\ und
\dq kein Licht\dq\ notwendig ist. Es ergibt sich also ein Empfangspuls.

Wird nun die Zeit gemessen zwischen Startsignal und empfangenem Puls, ist es möglich einen Rückschluss zu ziehen über die variable Strecke
zwischen Elektronik und Target. Die Zeitmessung unterliegt höchsten Anforderungen. Als Versinnbildlichung: Innert einer Nanosekunde
kann Licht eine Distanz von 30~cm zurücklegen. Wie eine solche Zeitmessung realisiert werden kann, erklärt das nächste Unterkapitel.

\subsection{Zeitmessung}

Damit ein direct \acrshort{tof}-Signal ausgewertet werden kann, benötigt es eine hochaufgelöste
Zeitmessung. Die Firma Texas Instruments bietet dafür mit dem TDC7200 eine gute Lösung
an. Laut Datenblatt \cite{ti2016tdc7200_datasheet} kann dieser bis zu 55~ps auflösen, mit einer
Standardabweichung von 35~ps. Setzt man nun die Lichtgeschwindigkeit sowie diese minimale
zeitliche Auflösung in eine Bewegungsgleichung ein, sollte der Baustein gemäss der Formel~\ref{eq:tdc_max_resolution}
folgende räumliche Auflösung erreichen:

\begin{equation}\label{eq:tdc_max_resolution}
        s_{min} = c \cdot t = 299.8 \cdot 10^6~\frac{m}{s} \cdot 55~ps = 16.5~mm
\end{equation}
\myequations{Maximale räumliche Auflösung des TDC7200}

Prinzipiell soll die \acrshort{tof}-Funktionalität in zwei Schritten in Betrieb genommen werden.
In einem ersten Teil soll es rein darum gehen, den TDC7200 \acrshort{ic} genauer
kennenzulernen.

Bemerkung: In der Realität wird eine solche Auflösung nicht erreichbar sein. Die Herausforderungen
bestehen beispielsweise darin, dass in der gesamten Signalkette diverse Verzögerungen vorhanden sind.
So spielen zum Beispiel das Einschalten der Laser-Diode, die Bandbreite des Verstärkers sowie
die Schaltzeit des Komparators eine grosse Rolle. Auch der Jitter der \acrshort{mcu} beim Ein- und Ausschalten
ihrer Ausgänge wird bei der minimalen Auflösung ins Gewicht fallen. Im Kapitel~\ref{sec:electrical_measurements}
wird mittels verschiedener Messungen dargestellt, wie nah an die vom TDC7200 vorgegebene Grenze herangekommen wird.

\subsection{Vorgehensweise}\label{sec:approach}

Wie bereits erläutert ist davon auszugehen, dass der gesamte Signalpfad eine Ungenauigkeit in die Messung einbringt,
welche weit ausserhalb derjenigen des TDC7200 liegt. Es bietet sich also an, dass diese im Vorfeld untersucht wird.
Dazu wird ein zweiter TDC7200 auf der Elektronik eingesetzt, welcher zur Messung eines rein elektrischen Signals
zuständig ist.

\begin{figure}[H]
    \centering
    \includegraphics[width=\textwidth]{diagrams/konzept_tdc_electrical.pdf}
    \caption{Konzept Betrieb \acrshort{tdc} rein elektrisch}\label{fig:konzept_tdc_electrical}
\end{figure}

In Abbildung~\ref{fig:konzept_tdc_electrical} sei skizziert, wie eine solche Schaltung aussehen könnte. Die gewellte,
rot markierte Linie versinnbildlicht dabei einen variablen Messpfad. Über einen Stecker können hier unterschiedlich lange
Kabel angeschlossen werden. Somit ist es möglich, die Messgenauigkeit des TDC7200 rein elektrisch nachzuweisen.

Folgende Punkte können mit einer solchen Erweiterung überprüft werden:

\begin{itemize}
    \item Messung Jitter der Ausgänge der verwendeten \acrshort{mcu}
    \item Konfiguration und Inbetriebnahme TDC7200 (z.B. verschiedene Messmodi)
    \item Auswertung der Resultate des TDC7200
\end{itemize}
