Beim Messverfahren, welches als \dq \acrlong{tof}\dq\ bekannt ist, handelt es sich um eine optische Distanzmessung.
Hierbei wird prinzipiell mit einem Sender Licht ausgesandt, welches von einem Empfänger detektiert wird. Zwischen dem
Senden und Empfangen des Signals kann eine Zeitmessung gemacht werden.

\begin{figure}[H]
    \centering
    \includegraphics[width=\textwidth]{diagrams/dToF_vs_iToF.pdf}
    \caption{Visualisierung von a: direct \acrshort{tof} und b: indirect \acrshort{tof}}\label{fig:dtof_vs_itof}
\end{figure}

Man unterscheidet zwei Arten von \acrshort{tof}: Direct \acrshort{tof} (dToF) und Indirect \acrshort{tof} (iToF). Bei dToF
wird einzig und alleine die Laufzeit zwischen Sende- und Empfangspuls ermittelt. Bei iToF wird (teilweise zusätzlich) zur
Laufzeitmessung auch die Phasenverschiebung eines modulierten Lichtsignales gemessen. Letztere Methode erlaubt es, höhere
Auflösungen zu erreichen, ist jedoch in der Auswertung und Signalaufbereitung ungleich komplexer.

Diese Arbeit beschränkt sich auf das direct \acrshort{tof} Verfahren.

Nachfolgend werden die theoretischen Grundlagen, vor allem in Hinblick auf das Auslegen einer geeigneten Optik, diskuttiert.

\pagebreak

\subsection{Photostrom}

Zur Berechnung des theoretisch zu erwartenden Photostrom wird von einer Distanz zur Wand von $10~m$ ausgegangen.

Der Laserstrahl gehe idealisiert mit $0\degree$ zur Wand und werde dort uniform Halbkugel-förmig gestreut. In der
Realität wird der Laser nicht mit $0\degree$ zur Wand gehen und die Streuung wird sich nicht uniform verteilen, sondern
in der Mitte stärker konzentriert sein.

Die Berechnung der empfangenen Strahlungsleistung, der Strahlungsintensität, dem Raumwinkel und dem Photostrom sind in
Formel~\ref{eq:pin}, \ref{eq:ie}, \ref{eq:omega} bzw. \ref{eq:iph} gezeigt.

\begin{equation}\label{eq:pin}
    P_{in} = E_e \cdot A = \frac{I_e}{r^2} \cdot A
\end{equation}
\myequations{Eintreffende Lichtleistung}

\begin{equation}\label{eq:ie}
    I_e = \frac{P_{out}}{\Omega}
\end{equation}
\myequations{Strahlungsintensität}

\begin{equation}\label{eq:omega}
    \Omega = \frac{4\cdot \pi \cdot 0.5}{d}
\end{equation}
\myequations{Raumwinkel}

\begin{equation}\label{eq:iph}
    I_{ph} = S \cdot P_{in}
\end{equation}
\myequations{Photostrom}

\subsubsection{Berechnung mit RLD94PZJ5 und BPV23NF}

Ersten Berechnungen wurden mit der Laserdiode RLD94PZJ5 \cite{rohm2020rld94pzj5_datasheet} und der Photodiode BPV23NF
\cite{vishay2024bpv23nf_datasheet} durchgeführt.

Die relevanten Werte aus den Datenblättern sind in Formel~\ref{eq:rld94pzj5_num} und \ref{eq:rbpv23nf_num} aufgelistet.

\begin{equation}\label{eq:rld94pzj5_num}
    P_{out} = 285~mW
\end{equation}
\myequations{Werte des RLD94PZJ5}

\begin{equation}\label{eq:rbpv23nf_num}
    \begin{split}
        A &= 4.4~mm^2\\
        S &= 0.6~\frac{A}{W}
    \end{split}
\end{equation}
\myequations{Werte des BPV23NF}

Diese Werte eingesetzt in Formel~\ref{eq:ie}, \ref{eq:pin} und \ref{eq:iph} ergibt die Resultate in
Formel~\ref{eq:rld94pzj5_rbpv23nf_num}.

\begin{equation}\label{eq:rld94pzj5_rbpv23nf_num}
    \begin{split}
        I_e    &= \frac{P_{out}}{\Omega} = \frac{285~mW}{4\cdot \pi \cdot 0.5~sr} = 45.4~\frac{mW}{sr}\\
        P_{in} &= \frac{I_e}{r^2} \cdot A = \frac{45.4~\frac{mW}{sr}}{(10~m)^2} \cdot 4.4~mm^2 = 2~nW\\
        I_{ph} &= S \cdot P_{in} = 0.6~\frac{A}{W} \cdot 2~nW = 1.2~nA
    \end{split}
\end{equation}
\myequations{Nummerische Resultate mit RLD94PZJ5 und BPV23NF}

\subsubsection{Berechnung mit RLD65NZX1 and NJL6401R-3}

Die Laserdiode RLD94PZJ5 hat im Bezug auf diese Projektarbeit zwei Nachteile: Sehr hohe Leistung, welche für das
menschliche Auge gefährlich werden kann und ein Wellenlängenbereich, der für das menschliche Auge nicht sichtbar ist.

Aus diesen Gründen wurde eine zweite Laserdiode evaluiert: RLD65NZX1 \cite{rohm2019rld65nzx1_datasheet}. Gepaart wird
sie mit der Photodiode NJL6401R-3 \cite{jrc2014njl6401r3_datasheet}. Die folgenden Berechnungen wurden basierend auf
diesen beiden Komponenten durchgeführt.

Die relevanten Werte aus den Datenblättern sind in Formel~\ref{eq:rld65nzx1_num} und \ref{eq:njl6401r3_num} aufgelistet.

\begin{equation}\label{eq:rld65nzx1_num}
    P_{out} = 10~mW
\end{equation}
\myequations{Werte des RLD65NZX1}

\begin{equation}\label{eq:njl6401r3_num}
    \begin{split}
        A &= 0.7~mm \cdot 0.7~mm = 0.49~mm^2\\
        S &= 0.42~\frac{A}{W}
    \end{split}
\end{equation}
\myequations{Werte des NJL6401R-3}

Diese Werte eingesetzt in Formel~\ref{eq:ie}, \ref{eq:pin} und \ref{eq:iph} ergibt die Resultate in
Formel~\ref{eq:rld65nzx1_njl6401r3_num}.

\begin{equation}\label{eq:rld65nzx1_njl6401r3_num}
    \begin{split}
        I_e    &= \frac{P_{out}}{\Omega} = \frac{10~mW}{\frac{4\cdot \pi \cdot 0.5}{d}} = 1.59~\frac{mW}{sr}\\
        P_{in} &= \frac{I_e}{r^2} \cdot A = \frac{1.59~\frac{mW}{sr}}{(10~m)^2} \cdot 0.49~mm^2 = 7.8~pW\\
        I_{ph} &= S \cdot P_{in} = 0.42~\frac{A}{W} \cdot 7.8~pW = 3.28~pA
    \end{split}
\end{equation}
\myequations{Nummerische Resultate mit RLD65NZX1 and NJL6401R-3}

\subsubsection{Berechnung mit Empfangs-Linse}

Auf Vorschlag der Dozenten wurden verschiedene Optiken aus einem Baukasten-System von QIOPTIQ ausprobiert.
Besonders vielversprechend erschien hierbei eine Linse mit einem Durchmesser von 17~mm bei einer Brennweite von
40~mm. Die Linse ist in Abbildung~\ref{fig:lens} dargestellt.

\begin{figure}[H]
    \centering
    \includegraphics[width=0.5\textwidth]{graphics/photo_lens.jpg}
    \caption{Linse von QIOPTIQ}\label{fig:lens}
\end{figure}

Eine Solche Linse vergrössert die effektive Fläche, auf welcher der Lichtstrom empfangen werden kann, was
eine höhere Empfangsleistung, sprich einen höheren Lichtstrom zur Folge hat.

Die Formel~\ref{eq:njl6401r3_lens} zeigt auf, um welchen Faktor die Fläche in etwa ansteigen wird, beim Einsatz einer
solchen Optik.

\begin{equation}\label{eq:njl6401r3_lens}
    A' = \frac{A_{L}}{A_{PD}} = \frac{(\frac{17~mm}{2})^2 \cdot \pi}{0.49~mm^2} = 463.2
\end{equation}
\myequations{Vergrösserung der Empfangsfläche durch Linse}

\subsubsection{Erwarteter Lichtstrom}
Nachfolgend werden zwei Szenarien betrachtet: Einmal wird wie bis anhin davon ausgegangen, dass sich das Licht ab dem Laser
in einem geraden Strahl ausbreitet, um danach halbkugelförmig und diffus am Target zu reflektieren.

Im zweiten Szenario wird versucht, die Abstrahlcharakteristik der Laser-Diode zu modellieren. Laut Datenblatt hat die
Diode in der Abstrahlebene zwei unterschiedliche Öffnungswinkel, was mit dem Raumwinkel einer rechteckigen Pyramide
annäherungsweise berechnet werden kann. Die Idee dahinter ist, dass einerseits die Abstrahlungscharakteristik
besser berücksichtigt wird, andererseits wird somit die eher in der Mitte konzentrierte Reflexion allenfalls besser abgebildet.

Bei beiden Szenarien soll sei die abgestrahlte Leistung der Laser-Diode höher als in der ersten Rechnung. Im gepulsten
Betrieb kann diese problemlos um Faktor 60 erhöht werden. Thermisch stellt dies kein Problem dar für die Diode, da
der Duty-Cycle sehr klein ist. Die Formel~\ref{eq:rld65nzx1_higher_output} zeigt die neu zu erwartete Abstrahl-Leistung
eines solchen Pulses. Die zur Umrechnung benötigte Effizient der Laser-Diode wird dem Datenblatt des RLD65NZX1 entnommen.
\cite{rohm2019rld65nzx1_datasheet}

\begin{equation}\label{eq:rld65nzx1_higher_output}
    P_{out}' = \eta \cdot I_{ld} = 0.85~\frac{W}{A} \cdot 0.6~A = 510~mW
\end{equation}
\myequations{Erhöhte Sendeleistung bei der Laser-Diode}

Bei einem Target, welches sich in 2~m Diszanz befindet, kann laut Formel~\ref{eq:photocurrent_scenario1} folgender
Lichtstrom erwartet werden:

\begin{equation}\label{eq:photocurrent_scenario1}
    \begin{split}
        P_{in} &= \frac{P_{out}}{\Omega \cdot r^2} \cdot A_{L} = \frac{510~mW}{4\pi \cdot 0.5~sr \cdot (2~m)^2} \cdot 227~mm^2 = 4.61~\mu W\\
        I_{ph} &= S \cdot P_{in} = 0.42~\frac{A}{W} \cdot 4.61~\mu W = 1.93~\mu A
    \end{split}
\end{equation}
\myequations{}

Ohne Optik kann die Laserdiode nicht gebündelt abstrahlen. Diese nicht ganz so ideale Abstrahlcharakteristik kann im
zweiten Szenario über den Raumwinkel einer Pyramide, wie er in Formel~\ref{eq:solid_angle_pyramid} definiert ist,
modelliert werden. Die typischen Abstrahlwinkel $\theta$ werden ebenfalls dem Datenblatt der Laser-Diode
entnommen werden. Hierbei ist zu beachten, dass im Datenblatt der Vollwinkel angegeben wird. In der Formel wird dieser
deswegen halbiert.

\begin{equation}\label{eq:solid_angle_pyramid}
    \Omega_{P} = 4 \cdot arcsin(sin(\theta_x) \cdot sin(\theta_y)) =  4 \cdot arcsin(sin(\frac{9~\degree}{2} ) \cdot sin(\frac{29~\degree}{2})) = 4.2~sr
\end{equation}
\myequations{Modell Abstrahlung Laserdiode}

Mit der Formel~\ref{eq:photocurrent_scenario2} wird erneut die zu erwartende Empfangsleistung sowie der dazugehörige
Photostrom errechnet. Der Unterschied zur vorherigen Berechnung liegt nun aber darin, dass der Raumwinkel auch bereits
vor der Reflexion berücksichtigt werden muss. Deswegen wird die Diszanz zum Target mit zwei multipliziert.

\begin{equation}\label{eq:photocurrent_scenario2}
    \begin{split}
        P_{in} &= \frac{P_{out}}{\Omega \cdot (2 \cdot r)^2} \cdot A_{L} = \frac{510~mW}{4.2~sr \cdot (2 \cdot 2~m)^2} \cdot 227~mm^2 = 1.72~\mu W\\
        I_{ph} &= S \cdot P_{in} = 0.42~\frac{A}{W} \cdot 1.72~\mu W = 724~nA
    \end{split}
\end{equation}
\myequations{}

\pagebreak

\subsection{Transimpedanzverstärker}

Bei einem \acrfull{tia} handelt es sich um \dots
