\setlength{\headheight}{26pt}

% Packages
\usepackage[left=25mm, right=25mm, top=35mm, bottom=25mm]{geometry} % defines whitespaces on the edges
\usepackage[bottom, hang]{footmisc}
\usepackage{hyperref}             % for \nameref and cite
\usepackage{footnotebackref}      % used for getting back from the footnote to the main text
\usepackage{enumitem}
\usepackage{graphicx}
\usepackage{fancyhdr}
\usepackage{lastpage}
\usepackage[export]{adjustbox}
\usepackage{float}
\usepackage[ngerman]{babel}
\usepackage[ngerman]{datetime}
\usepackage{apacite}
\usepackage[utf8]{inputenc}
\usepackage{listings}
\usepackage{xcolor}
\usepackage{pdfpages}
\usepackage{url}
\usepackage{outlines}
\usepackage{tocloft}
\usepackage[utf8]{inputenc}
\usepackage{amsmath}
\usepackage{mathtools}
\usepackage[acronym]{glossaries}
\usepackage{titlesec}             % for \titleformat
\usepackage{textcomp}             % for \degree sign
\usepackage{gensymb}              % for \degree sign
\usepackage{csvsimple}            % for using csv-files for tables
\usepackage{lscape}               % for landscape mode
\usepackage{parskip}              % inserts space after paragraph automatically
\usepackage{svg}                  % used to include svg with the help of inkscape which converts svg to pdf
\usepackage{wasysym}              % for \diameter symbol
\usepackage{xfrac}                % for \sfrac

\frenchspacing  % Use same space size between words and between sentences

% change section titles' font size
\titleformat*{\section}{\huge\bfseries}
\titleformat*{\subsection}{\LARGE\bfseries}
\titleformat*{\subsubsection}{\Large\bfseries}
\titleformat{\paragraph}[hang]{\large\bfseries}{\theparagraph}{1em}{}
\titleformat{\subparagraph}[hang]{\normalsize\bfseries}{\thesubparagraph}{1em}{}

\definecolor{codegreen}{rgb}{0,0.6,0}
\definecolor{codegray}{rgb}{0.5,0.5,0.5}
\definecolor{codepurple}{rgb}{0.58,0,0.82}
\definecolor{codeorange}{rgb}{1,0.5,0.15}
\definecolor{backcolour}{rgb}{0.9,0.9,0.9}

\lstdefinestyle{mystyle}{
    backgroundcolor=\color{backcolour},
    commentstyle=\color{codegreen},
    keywordstyle=\color{magenta},
    numberstyle=\tiny\color{codegray},
    stringstyle=\color{codepurple},
    basicstyle=\ttfamily\footnotesize,
    breakatwhitespace=false,
    breaklines=true,
    captionpos=b,
    keepspaces=true,
    numbers=left,
    numbersep=5pt,
    showspaces=false,
    showstringspaces=false,
    showtabs=false,
    tabsize=4
}
\lstset{style=mystyle}

\lstset{
    literate={~} {$\sim$}{1}
}

\lstset{%
    breaklines=true,
    breakatwhitespace=true,
}

\DeclarePairedDelimiter\abs{\lvert}{\rvert} % make scalable absolute stripes
\DeclarePairedDelimiter\parenth{(}{)} % make scalable parentheses

\let\phi\varphi{} % change style of \phi sign

\setlength{\parindent}{0mm} % disable paragraph indent

\newdateformat{mydate}{\THEDAY{. }\monthnamengerman[\THEMONTH] \THEYEAR}

\renewcommand{\listfigurename}{}
\renewcommand\contentsname{Inhaltsverzeichnis}
\renewcommand\lstlistingname{Code}

\makeglossaries{}

% Header/Footer Setting
\setlength\footnotemargin{15pt}
\pagestyle{fancy}
\fancyhf{}
\renewcommand{\footrulewidth}{0.4pt} % footer line
\rhead{\textbf{\vUniversity}\\\vModule}
\lhead{\textbf{\vTitle}\\
    Projektarbeit}
\lfoot{\vAuthorFirstName{} \vAuthorLastName}
\cfoot{\mydate\today}
\rfoot{S.~\thepage~/~\pageref{LastPage}}

% Redefine the plain page style, otherwise there is no header and footer for chapter pages
\fancypagestyle{plain}{%
    \fancyhf{}
    \renewcommand{\footrulewidth}{0.4pt} % footer line
    \rhead{\textbf{\vUniversity}\\\vModule}
    \lhead{\textbf{\vTitle}\\
        Projektarbeit}
    \lfoot{\vAuthorFirstName{} \vAuthorLastName}
    \cfoot{\mydate\today}
    \rfoot{S.~\thepage~/~\pageref{LastPage}}
}

\bibliographystyle{apacite}

% Settings for the equation list
\newcommand{\listequationsname}{}
\newlistof{myequations}{equ}{\listequationsname}
\renewcommand{\cftmyequationsaftersnum}{\hfill}
\renewcommand{\cftmyequationspresnum}{\hfill}
\setlength{\cftmyequationsnumwidth}{3.5em}
\newcommand{\myequations}[1]{%
\addcontentsline{equ}{myequations}{\protect\numberline{\theequation}#1}\par}

\newcommand{\mytable}[4]
{
    \centering
    \begin{tabular}{#1}\hline%
        #2 \\ \hline
        \csvreader[
            separator=semicolon,
            head to column names,
            late after line=\\,
        ]{#4}{}{#3}
        \hline
    \end{tabular}
}
