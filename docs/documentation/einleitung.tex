In diesem Kapitel wird eine Einleitung in die Thematik der vorliegenden Abschlussarbeit gegeben.

Verfasst wird diese Arbeit von Timon Burkard und Matthias Schär. Betreuend stehen die Dozenten Prof. Guido Keel und
Michael Lehman zur Seite.

\subsection{Thema}
Die Arbeit befasst sich mit der Entwicklung einer Demonstrator-Elektronik für die Veranschaulichung von optischer \acrfull{tof} zur
Distanzmessung.

Es wird ein Mess-\acrshort{ic}, der TDC7200 von der Firma Texas Instruments, in Betrieb genommen. Dies geschieht in zwei
Schritten: Zuerst sollen rein elektrische Messungen ermöglicht werden, um beispielsweise die Länge von angeschlossenen
Kabeln zu bestimmen. Weiter soll sich auf dem Demonstrator zudem die nötige Optoelektronik finden, welche für eine
optische Distanzmessung vonnöten ist.

\subsection{Zielsetzung}
Die Studierenden haben sich zum Ziel gesetzt, eine Elektronik zu entwickeln, mit der sich die Versuche durchführen
lassen. Dabei soll die Ansteuerung der Mess-\acrshort{ic}s sowie das Auslegen der Optoelektronik ebenfalls weitgehend
selbstständig geschehen.

Es soll ein Demonstrator entstehen, mit welchem es möglich ist, unter Einsatz des TDC7200 unterschiedliche Kabellängen von
Kupferkabeln zu bestimmen. Nach dem dieses Ziel erreicht wird, soll es weiter möglich sein, optische Messungen mit dem
selben Demonstrator durchzuführen. Die Studierenden haben sich hierbei zum Ziel gesetzt, dass die Messungen mit einer Wand
als Target bis zu einer Distanz von 10~m funktionieren. Weiter sollte eine räumliche Auflösung von etwa 30~cm erreicht werden.

\subsection{Gliederung}
Nachfolgend sei ein kurzer Überblick über die folgenden Kapitel im Rest dieser Arbeit gegeben:


\begin{itemize}
    \item \textbf{Kapitel~\ref{sec:theory}} befasst sich mit den theoretischen Grundlagen, welche für die Entwicklung des
          Demonstrators erarbeitet wurden.

    \item \textbf{Kapitel~\ref{sec:concept}} erklärt die groben Konzepte und Herausforderungen, welche es zu bewältigen gibt.
          Hierbei werden insbesondere die zu entwickelnde Elektronik und die zeitlichen Anforderungen an den \acrshort{tdc} erfasst.

    \item \textbf{Kapitel~\ref{sec:simulations}} beinhaltet informationen zu den durchgeführten Simulationen, welche vor
          dem Design aber auch während der Inbetriebnahme des Demonstrators angestellt werden.

    \item \textbf{Kapitel~\ref{sec:realisation}} erläutert die Umsetzung in die Realität. Hier werden Schaltungen sowie
          der Herstellungsprozess präsentiert. Zudem wird erklärt, was an Software-Entwicklung notwendig ist, um einen
          solchen Demonstrator zu realisieren.

    \item \textbf{Kapitel~\ref{sec:measurements}} dokumentiert sämtliche angestellten Messungen sowie deren Resultate. Hier
          ist ersichtlich, inwiefern die initiale Zielsetzung erreicht wurde.

    \item \textbf{Kapitel~\ref{sec:conclusion}} erfasst abschliessende Gedanken der Studierenden und bildet ein Fazit der Arbeit.
          Es wird eine Aussicht auf mögliche Erweiterungen und Verbesserungen dargeboten.
\end{itemize}
