\documentclass[11pt,a4paper,hidelinks]{article}

\setlength{\headheight}{26pt}

% Packages
\usepackage[left=25mm, right=25mm, top=35mm, bottom=25mm]{geometry} % defines whitespaces on the edges
\usepackage[bottom, hang]{footmisc}
\usepackage{hyperref}             % for \nameref and cite
\usepackage{footnotebackref}      % used for getting back from the footnote to the main text
\usepackage{enumitem}
\usepackage{graphicx}
\usepackage{fancyhdr}
\usepackage{lastpage}
\usepackage[export]{adjustbox}
\usepackage{float}
\usepackage[ngerman]{babel}
\usepackage[ngerman]{datetime}
\usepackage{apacite}
\usepackage[utf8]{inputenc}
\usepackage{listings}
\usepackage{xcolor}
\usepackage{pdfpages}
\usepackage{url}
\usepackage{outlines}
\usepackage{tocloft}
\usepackage[utf8]{inputenc}
\usepackage{amsmath}
\usepackage{mathtools}
\usepackage[acronym]{glossaries}
\usepackage{titlesec}             % for \titleformat
\usepackage{textcomp}             % for \degree sign
\usepackage{gensymb}              % for \degree sign
\usepackage{csvsimple}            % for using csv-files for tables
\usepackage{lscape}               % for landscape mode
\usepackage{parskip}              % inserts space after paragraph automatically
\usepackage{svg}                  % used to include svg with the help of inkscape which converts svg to pdf
\usepackage{wasysym}              % for \diameter symbol
\usepackage{xfrac}                % for \sfrac

\frenchspacing  % Use same space size between words and between sentences

% change section titles' font size
\titleformat*{\section}{\huge\bfseries}
\titleformat*{\subsection}{\LARGE\bfseries}
\titleformat*{\subsubsection}{\Large\bfseries}
\titleformat{\paragraph}[hang]{\large\bfseries}{\theparagraph}{1em}{}
\titleformat{\subparagraph}[hang]{\normalsize\bfseries}{\thesubparagraph}{1em}{}

\definecolor{codegreen}{rgb}{0,0.6,0}
\definecolor{codegray}{rgb}{0.5,0.5,0.5}
\definecolor{codepurple}{rgb}{0.58,0,0.82}
\definecolor{codeorange}{rgb}{1,0.5,0.15}
\definecolor{backcolour}{rgb}{0.9,0.9,0.9}

\lstdefinestyle{mystyle}{
    backgroundcolor=\color{backcolour},
    commentstyle=\color{codegreen},
    keywordstyle=\color{magenta},
    numberstyle=\tiny\color{codegray},
    stringstyle=\color{codepurple},
    basicstyle=\ttfamily\footnotesize,
    breakatwhitespace=false,
    breaklines=true,
    captionpos=b,
    keepspaces=true,
    numbers=left,
    numbersep=5pt,
    showspaces=false,
    showstringspaces=false,
    showtabs=false,
    tabsize=4
}
\lstset{style=mystyle}

\lstset{
    literate={~} {$\sim$}{1}
}

\lstset{%
    breaklines=true,
    breakatwhitespace=true,
}

\DeclarePairedDelimiter\abs{\lvert}{\rvert} % make scalable absolute stripes
\DeclarePairedDelimiter\parenth{(}{)} % make scalable parentheses

\let\phi\varphi{} % change style of \phi sign

\setlength{\parindent}{0mm} % disable paragraph indent

\newdateformat{mydate}{\THEDAY{. }\monthnamengerman[\THEMONTH] \THEYEAR}

\renewcommand{\listfigurename}{}
\renewcommand\contentsname{Inhaltsverzeichnis}
\renewcommand\lstlistingname{Code}

\makeglossaries{}

% Header/Footer Setting
\setlength\footnotemargin{15pt}
\pagestyle{fancy}
\fancyhf{}
\renewcommand{\footrulewidth}{0.4pt} % footer line
\rhead{\textbf{\vUniversity}\\\vModule}
\lhead{\textbf{\vTitle}\\
    Projektarbeit}
\lfoot{\vAuthorFirstName{} \vAuthorLastName}
\cfoot{\mydate\today}
\rfoot{S.~\thepage~/~\pageref{LastPage}}

% Redefine the plain page style, otherwise there is no header and footer for chapter pages
\fancypagestyle{plain}{%
    \fancyhf{}
    \renewcommand{\footrulewidth}{0.4pt} % footer line
    \rhead{\textbf{\vUniversity}\\\vModule}
    \lhead{\textbf{\vTitle}\\
        Projektarbeit}
    \lfoot{\vAuthorFirstName{} \vAuthorLastName}
    \cfoot{\mydate\today}
    \rfoot{S.~\thepage~/~\pageref{LastPage}}
}

\bibliographystyle{apacite}

% Settings for the equation list
\newcommand{\listequationsname}{}
\newlistof{myequations}{equ}{\listequationsname}
\renewcommand{\cftmyequationsaftersnum}{\hfill}
\renewcommand{\cftmyequationspresnum}{\hfill}
\setlength{\cftmyequationsnumwidth}{3.5em}
\newcommand{\myequations}[1]{%
\addcontentsline{equ}{myequations}{\protect\numberline{\theequation}#1}\par}

\newcommand{\mytable}[4]
{
    \centering
    \begin{tabular}{#1}\hline%
        #2 \\ \hline
        \csvreader[
            separator=semicolon,
            head to column names,
            late after line=\\,
        ]{#4}{}{#3}
        \hline
    \end{tabular}
}


% Variables
\newcommand{\vTitle}{DIY Optische ToF Distanzmessung}
\newcommand{\vModule}{CAS Sensorik und Sensor Signal Conditioning}
\newcommand{\vAuthorFirstName}{Matthias Schär,}
\newcommand{\vAuthorLastName}{Timon Burkard}
\newcommand{\vUniversity}{OST -- Ostschweizer Fachhochschule}
\newcommand{\vDegree}{CAS Sensorik und Sensor Signal Conditioning}
\newcommand{\vSemester}{HS24}
\newcommand{\vProfessor}{Prof. Guido Keel, Michael Lehmann}
\newcommand{\vCity}{Rapperswil}
\newcommand{\vAbstract}{Die vorliegende Projektarbeit befasst sich mit der Entwicklung eines\dots}

% Acronyms
\newacronym{pcb}{PCB}{Printed Circuit Board}
\newacronym{tof}{ToF}{Time of Flight}
\newacronym{diy}{DIY}{Do It Yourself}
\newacronym{tdc}{TDC}{Time to Digital Converter}
\newacronym{tia}{TIA}{Trans-Impedance Amplifier}
\newacronym{mcu}{MCU}{Microcontroller Unit}
\newacronym{cpu}{CPU}{Central Processing Unit}
\newacronym{hal}{HAL}{Hardware Abstraction Layer}
\newacronym{gpio}{GPIO}{General Purpose Input/Output}
\newacronym{uart}{UART}{Universal Asynchronous Receiver/Transmitter}
\newacronym{dso}{DSO}{Digital Storage Oscilloscope}

\begin{document}

\title{\begin{huge}\textbf{\vTitle}\end{huge}\\
\ \\ \vModule}
\author{\\\\\textbf{\vAuthorFirstName{} \vAuthorLastName}\\
    \vUniversity{}\\
    \\\\}

\date{\mydate\today}
\maketitle\thispagestyle{empty}  % removes footer from first page

\vspace{70pt}

\begin{figure}[H]
    \centering
    \begin{minipage}[b]{0.35\textwidth}
        \includesvg[width=\textwidth]{graphics/OST_Logo.svg}\label{fig:OST_Logo}
    \end{minipage}
\end{figure}

\pagebreak

\thispagestyle{empty}

\textbf{}
\vspace{5mm}

\begin{flushleft}
    \textbf{\large{\vModule{} an der \vUniversity{}}}
\end{flushleft}

\begin{flushleft}
    \begin{small}
        \begin{tabular}{@{}lll}
            \\
            \textbf{Titel}                 & & \textbf{\vTitle}\\
            \\
            \textbf{Diplomandin/Diplomand} & & \textbf{\vAuthorFirstName{} \vAuthorLastName}\\
            \\
            \textbf{Studiengang}           & & \textbf{\vDegree}\\
            \\
            \textbf{Semester}              & & \textbf{\vSemester}\\
            \\
            \textbf{Dozentin/Dozent}       & & \textbf{\vProfessor}\\
        \end{tabular}
    \end{small}
\end{flushleft}

\vspace{10mm}

% abstracts
\begin{small}
    \textbf{Abstract}

    \vAbstract{}
    \vspace{15mm}
\end{small}

% copyright
\begin{flushleft}
    \begin{small}
        \vCity, \date{\mydate\today} \\
        \textbf{\textcopyright\hspace{1mm}\vAuthorFirstName{} \vAuthorLastName, \vUniversity{}}
    \end{small}
\end{flushleft}

\mbox{}
\vfill

\pagebreak

\section*{Inhaltsverzeichnis}
\vspace{-25pt}
\renewcommand*\contentsname{}
\tableofcontents

\pagebreak

\section*{Abkürzungsverzeichnis}
\vspace{-25pt}
\printglossary[type=\acronymtype,title={}]

\pagebreak

\section*{Abbildungsverzeichnis}
\vspace{-25pt}
\renewcommand{\listfigurename}{}
\listoffigures

\pagebreak

\section*{Formelverzeichnis}
\vspace{-25pt}
\listofmyequations{}

\pagebreak

\section*{Tabellenverzeichnis}
\vspace{-25pt}
\renewcommand{\listtablename}{}
\listoftables

\pagebreak

\section*{Codeverzeichnis}
\vspace{-25pt}
\renewcommand{\lstlistlistingname}{}
\lstlistoflistings{}

\pagebreak


%%%%%%%%%%%%%%%%%%%%%% CONTENT START %%%%%%%%%%%%%%%%%%%%%%

\section{Einleitung}

Bei dieser Projektarbeit geht es darum ein \acrfull{diy} optisches \acrfull{tof} Distanzmesssystem aufzubauen. Dazu soll
ein \acrfull{tdc} verwendet werden.

\dots

\pagebreak

\section{Theorie}

\subsection{Time of Flight}

Bei \acrshort{tof} handelt es sich um \dots

\pagebreak

\subsection{Photostrom}

Zur Berechnung des theoretisch zu erwartenden Photostrom wird von einer Distanz zur Wand von $10~m$ ausgegangen.

Der Laserstrahl gehe idealisiert mit $0\degree$ zur Wand und werde dort uniform Halbkugel-förmig gestreut. In der
Realität wird sich die Streuung nicht uniform verteilen, sondern in der Mitte stärker konzentriert sein. Die folgende
Berechnung zeigt also den worst case.

Die Berechnung der empfangenen Strahlungsleistung, der Strahlungsintensität, dem Raumwinkel und dem Photostrom sind in
Formel~\ref{eq:pin}, \ref{eq:ie}, \ref{eq:omega} bzw. \ref{eq:iph} gezeigt.

\begin{equation}\label{eq:pin}
    P_{in} = E_e \cdot A = \frac{I_e}{r^2} \cdot A
\end{equation}
\myequations{Eintreffende Lichtleistung}

\begin{equation}\label{eq:ie}
    I_e = \frac{P_{out}}{\Omega}
\end{equation}
\myequations{Strahlungsintensität}

\begin{equation}\label{eq:omega}
    \Omega = \frac{4\cdot \pi \cdot 0.5}{d}
\end{equation}
\myequations{Raumwinkel}

\begin{equation}\label{eq:iph}
    I_{ph} = S \cdot P_{in}
\end{equation}
\myequations{Photostrom}

\subsubsection{Berechnung mit RLD94PZJ5 und BPV23NF}

Ersten Berechnungen wurden mit der Laserdiode RLD94PZJ5 \cite{rohm2020rld94pzj5_datasheet} und der Photodiode BPV23NF
\cite{vishay2024bpv23nf_datasheet} durchgeführt.

Die relevanten Werte aus den Datenblättern sind in Formel~\ref{eq:rld94pzj5_num} und \ref{eq:rbpv23nf_num} aufgelistet.

\begin{equation}\label{eq:rld94pzj5_num}
    P_{out} = 285~mW
\end{equation}
\myequations{Werte des RLD94PZJ5}

\begin{equation}\label{eq:rbpv23nf_num}
    \begin{split}
        A &= 4.4~mm^2\\
        S &= 0.6~\frac{A}{W}
    \end{split}
\end{equation}
\myequations{Werte des BPV23NF}

Diese Werte eingesetzt in Formel~\ref{eq:ie}, \ref{eq:pin} und \ref{eq:iph} ergibt die Resultate in
Formel~\ref{eq:rld94pzj5_rbpv23nf_num}.

\begin{equation}\label{eq:rld94pzj5_rbpv23nf_num}
    \begin{split}
        I_e    &= \frac{P_{out}}{\Omega} = \frac{285~mW}{\frac{4\cdot \pi \cdot 0.5}{d}} = \frac{285~mW}{\frac{4\cdot \pi \cdot 0.5}{10~m}} = 45~\frac{mW}{sr}\\
        P_{in} &= \frac{I_e}{r^2} \cdot A = 45~\frac{mW}{sr} \cdot 4.4~mm^2 = 2~nW\\
        I_{ph} &= S \cdot P_{in} = 0.6~\frac{A}{W} \cdot 2~nW = 1.2~nA
    \end{split}
\end{equation}
\myequations{Nummerische Resultate mit RLD94PZJ5 und BPV23NF}

\subsubsection{Berechnung mit RLD65NZX1 and NJL6401R-3}

Die Laserdiode RLD94PZJ5 hat im Bezug auf diese Projektarbeit zwei Nachteile: Sehr hohe Leistung, welche für das
menschliche Auge gefährlich werden kann und ein Wellenlängenbereich, der für das menschliche Auge nicht sichtbar ist.

Aus diesen Gründen wurde eine zweite Laserdiode evaluiert: RLD65NZX1 \cite{rohm2019rld65nzx1_datasheet}. Gepaart wird
sie mit der Photodiode NJL6401R-3 \cite{jrc2014njl6401r3_datasheet}. Die folgenden Berechnungen wurden basierend auf
diesen beiden Komponenten durchgeführt.

Die relevanten Werte aus den Datenblättern sind in Formel~\ref{eq:rld65nzx1_num} und \ref{eq:njl6401r3_num} aufgelistet.

\begin{equation}\label{eq:rld65nzx1_num}
    P_{out} = 10~mW
\end{equation}
\myequations{Werte des RLD65NZX1}

\begin{equation}\label{eq:njl6401r3_num}
    \begin{split}
        A &= 0.7~mm \cdot 0.7~mm = 0.49~mm^2\\
        S &= 0.42~\frac{A}{W}
    \end{split}
\end{equation}
\myequations{Werte des NJL6401R-3}

Diese Werte eingesetzt in Formel~\ref{eq:ie}, \ref{eq:pin} und \ref{eq:iph} ergibt die Resultate in
Formel~\ref{eq:rld65nzx1_njl6401r3_num}.

\begin{equation}\label{eq:rld65nzx1_njl6401r3_num}
    \begin{split}
        I_e    &= \frac{P_{out}}{\Omega} = \frac{10~mW}{\frac{4\cdot \pi \cdot 0.5}{d}} = \frac{10~mW}{\frac{4\cdot \pi \cdot 0.5}{10~m}} = 1.6~\frac{mW}{sr}\\
        P_{in} &= \frac{I_e}{r^2} \cdot A = 45~\frac{mW}{sr} \cdot 0.49~mm^2 = 8~pW\\
        I_{ph} &= S \cdot P_{in} = 0.42~\frac{A}{W} \cdot 8~pW = 3.3~pA
    \end{split}
\end{equation}
\myequations{Nummerische Resultate mit RLD65NZX1 and NJL6401R-3}

\pagebreak

\subsection{Transimpedanzverstärker}

Bei einem \acrfull{tia} handelt es sich um \dots

\pagebreak

\section{Umsetzung}

\subsection{Firmware}

Der selbst entwickelte Firmware-Treiber für den \acrshort{tdc} befindet im Anhang Kapitel \ref{sec:tdc_driver}.

\pagebreak

\subsection{Schaltungen}

\subsubsection{Selective Input Voltage}

Abbildung~\ref{fig:selective_input_voltage} zeigt die Beschaltung zur Selektion der Speisung.

\begin{figure}[H]
    \centering
    \includegraphics[page=2, trim=80 590 750 50, clip, width=0.9\textwidth]{attachments/schematic.pdf}
    \caption{Selective Input Voltage}\label{fig:selective_input_voltage}
\end{figure}

Für die Speisung des Nucleo-Boards bestehen die folgenden Möglichkeiten:

\begin{itemize}
    \item 5V von USB-Buchse
    \item 5V von externem Power-Supply (JP1 + JP2)
    \item 12V von externem Power-Supply (JP3)
\end{itemize}

Siehe dazu auch Kapitel~\ref{sec:schematic_nucleo}.

Für die Speisung der 5V-Elektronik bestehen die folgenden Möglichkeiten:

\begin{itemize}
    \item 5V von Nucleo-Board (JP1)
    \item 5V von externem Power-Supply (JP2)
    \item 12V von externem Power-Supply via Nucleo-Board (JP1 + JP3)
\end{itemize}

Für die Speisung der Photodiode bestehen die folgenden Möglichkeiten:

\begin{itemize}
    \item 5V von 5V-Elektronik (SW2 Position 3)
    \item 12V von externem Power-Supply (SW2 Position 1)
\end{itemize}

Siehe dazu auch Kapitel~\ref{sec:schematic_photo_receiver}.

\subsubsection{Nucleo Board}\label{sec:schematic_nucleo}

Die Beschaltung des NUCLEO-F042K6 Boards \cite{st2024nucleof042k6_usermanual} ist in Abbildung~\ref{fig:nucleo_board}
gezeigt.

\begin{figure}[H]
    \centering
    \includegraphics[page=2, trim=530 580 300 50, clip, width=0.9\textwidth]{attachments/schematic.pdf}
    \caption{Nucleo Board}\label{fig:nucleo_board}
\end{figure}

\subsubsection{TDC Electrical Signal}

Die Beschaltung des TDC7200 \cite{ti2016tdc7200_datasheet} für den elektrischen Teil ist in
Abbildung~\ref{fig:tdc_ele_signal} gezeigt.

\begin{figure}[H]
    \centering
    \includegraphics[page=2, trim=80 330 750 310, clip, width=0.9\textwidth]{attachments/schematic.pdf}
    \caption{TDC Electrical Signal}\label{fig:tdc_ele_signal}
\end{figure}

\subsubsection{TDC Optical Signal}

Die Beschaltung des TDC7200 \cite{ti2016tdc7200_datasheet} für den optischen Teil ist in
Abbildung~\ref{fig:tdc_opt_signal} gezeigt.

\begin{figure}[H]
    \centering
    \includegraphics[page=2, trim=530 330 300 310, clip, width=0.9\textwidth]{attachments/schematic.pdf}
    \caption{TDC Optical Signal}\label{fig:tdc_opt_signal}
\end{figure}

\subsubsection{Oscillator For TDCs}

Die Beschaltung des Oszillators für die beiden \acrshort{tdc} ist in Abbildung~\ref{fig:oscillator_tdc} gezeigt.

\begin{figure}[H]
    \centering
    \includegraphics[page=2, trim=80 90 930 550, clip, width=0.45\textwidth]{attachments/schematic.pdf}
    \caption{Oscillator for TDCs}\label{fig:oscillator_tdc}
\end{figure}

\subsubsection{Power Supply Separation}

Für die Beschaltung der Photodiode, inkl. \acrshort{tia} und Komparator, macht es Sinn eine Spannungsversorgung mit
möglichst wenig Rauschen zu haben.

Dazu wurde die Separierung vorgenommen, welche in Abbildung~\ref{fig:power_supply_separation} dargestellt ist.

\begin{figure}[H]
    \centering
    \includegraphics[page=2, trim=260 90 640 550, clip, width=0.7\textwidth]{attachments/schematic.pdf}
    \caption{Power Supply Separation}\label{fig:power_supply_separation}
\end{figure}

\subsubsection{Laser Driver}

Die Laser Diode RLD65NZX1 \cite{rohm2019rld65nzx1_datasheet} wird mittels Lasertreiber LMG1025-Q1
\cite{ti2024lmg1025q1_datasheet} und NexFET \cite{ti2016csd17578q3a_datasheet} angesteuert. Für die Generierung eines
kurzen Pulses (0.5 \dots 100~ns) wurde mittels Hochpass und AND-Gatter \cite{diodes202074lvc1g08q_datasheet}
implementiert. Siehe dazu Abbildung~\ref{fig:laser_driver}.

\begin{figure}[H]
    \centering
    \includegraphics[page=3, trim=100 520 550 60, clip, width=0.9\textwidth]{attachments/schematic.pdf}
    \caption{Laser Driver}\label{fig:laser_driver}
\end{figure}

\subsubsection{Photo Receiver}\label{sec:schematic_photo_receiver}

Um den Photostrom der Photodiode NJL6401R \cite{jrc2014njl6401r3_datasheet} zu verstärken und in eine Spannung
umzuwandeln, wurde mit dem Operationsverstärker OPA858 \cite{ti2018opa858_datasheet} ein Transimpedanzverstärker
aufgebaut. Der Ausgangs des Transimpedanzverstärkers geht auf den Komparator TLV3501 \cite{ti2016tlv3501_datasheet},
um das STOP-Signal für den \acrshort{tdc} zu generieren. Siehe dazu Abbildung~\ref{fig:photo_receiver}.

\begin{figure}[H]
    \centering
    \includegraphics[page=3, trim=100 240 600 340, clip, width=0.9\textwidth]{attachments/schematic.pdf}
    \caption{Photo Receiver}\label{fig:photo_receiver}
\end{figure}

\subsubsection{Decoupling Capacitors}

Die Beschaltung der Entkopplungs-Kondensatoren ist in Abbildung~\ref{fig:decoupling_capacitors} dargestellt.

\begin{figure}[H]
    \centering
    \includegraphics[page=3, trim=100 60 650 630, clip, width=0.9\textwidth]{attachments/schematic.pdf}
    \caption{Decoupling Capacitors}\label{fig:decoupling_capacitors}
\end{figure}

\pagebreak

\subsection{Layout}

In diesem Kapitel werden die \acrshort{pcb}-Layouts dokumentiert.

\subsubsection{Kupfer-Layer}

\begin{figure}[H]
    \centering
    \includegraphics[trim=130 220 450 80, clip, width=0.6\textwidth]{attachments/pcb_F_Cu.pdf}
    \caption{PCB Layout Top}\label{fig:pcb_f_cu}
\end{figure}

\begin{figure}[H]
    \centering
    \includegraphics[trim=130 220 450 80, clip, width=0.6\textwidth]{attachments/pcb_B_Cu.pdf}
    \caption{PCB Layout Bottom}\label{fig:pcb_b_cu}
\end{figure}

\begin{figure}[H]
    \centering
    \includegraphics[trim=130 220 450 80, clip, width=0.6\textwidth]{attachments/pcb_In1_Cu.pdf}
    \caption{PCB Layout Innen 1}\label{fig:pcb_in1_cu}
\end{figure}

\begin{figure}[H]
    \centering
    \includegraphics[trim=130 220 450 80, clip, width=0.6\textwidth]{attachments/pcb_In2_Cu.pdf}
    \caption{PCB Layout Innen 2}\label{fig:pcb_in2_cu}
\end{figure}

\subsubsection{Komponenten-Platzierung}

\begin{figure}[H]
    \centering
    \includegraphics[page=1, trim=120 220 450 80, clip, width=0.6\textwidth]{attachments/pcb_placement.pdf}
    \caption{PCB Komponenten-Platzierung Top}\label{fig:pcb_placement_1}
\end{figure}

\begin{figure}[H]
    \centering
    \includegraphics[page=2, trim=450 220 120 80, clip, width=0.6\textwidth]{attachments/pcb_placement.pdf}
    \caption{PCB Komponenten-Platzierung Bottom}\label{fig:pcb_placement_2}
\end{figure}

\pagebreak

\subsection{3D View}

\begin{figure}[H]
    \centering
    \includegraphics[width=0.6\textwidth]{graphics/3d_top.png}
    \caption{3D View Top}\label{fig:3d_top}
\end{figure}

\begin{figure}[H]
    \centering
    \includegraphics[width=0.6\textwidth]{graphics/3d_bottom.png}
    \caption{3D View Bottom}\label{fig:3d_bottom}
\end{figure}

\begin{landscape}

\subsection{Komponenten}

\begin{table}[H]
    \scriptsize
    \mytable
        {|l|l|l|l|l|l|}
        {\textbf{Reference} & \textbf{Value} & \textbf{Datasheet} & \textbf{Footprint} & \textbf{Qty} & \textbf{DNP}}
        {\Reference & \Value & \Datasheet & \Footprint & \Qty & \DNP}
        {tables/bom.csv}
    \caption{Bill of Material}\label{tab:bom}
\end{table}

\end{landscape}

\pagebreak

\section{Simulationen}

\subsection{Laser Treiber}
\subsection{Transimpedanzverstärker}

\pagebreak

\section{Messungen}

In diesem Kapitel werden die Messresultate dokumentiert.

Die verwendeten Python-Skripte zur Berechnung der statistischen Grössen und zum Plotten der Diagramme befinden sich im
Anhang Kapitel \ref{sec:python_analyze}.

Das verwendete \acrfull{dso} ist ein Rohde \& Schwarz RTB2004 1.25 GSa/s.

\subsection{Elektrische Messungen}

In diesem Teilkapitel werden die Messresultate dokumentiert, welche rein elektrisch (also ohne optischen Teil) erfasst
wurde.

Die Zeitmessungen werden von \lstinline|IC1| (siehe Abbildung~\ref{fig:tdc_ele_signal}) durchgeführt und von der
Firmware, welche auf dem Nucleo Board \lstinline|U1| (siehe Abbildung~\ref{fig:nucleo_board}) getriggert und ausgelesen.

\subsubsection{GPIO Toggle mit HAL}\label{sec:gpio_toggle_with_hal}

Als erstes wird gemessen, wie lange es für die \acrshort{cpu} der \acrshort{mcu} dauert mittels \acrfull{hal} - Library
\cite{st2020stm32f0_hal} zwei \acrshort{gpio}-Pins zu schalten.

In Code~\ref{code:gpio_toggle_with_hal} ist die Firmware-Implementation dazu gezeigt.

\lstinputlisting[language={C}, label={code:gpio_toggle_with_hal}, caption={\acrshort{gpio} Toggle mit \acrshort{hal}}]{sourcecode/gpio_toggle_with_hal.c}

Vom \acrshort{tdc} wird also die Zeit gemessen zwischen Zeile 15 und 16 in Code~\ref{code:gpio_toggle_with_hal}. Dazu
wird \lstinline|SW1| mit \lstinline|stop_ele| verbunden und \lstinline|J3| kurzgeschlossen (siehe
Abbildung~\ref{fig:tdc_ele_signal}).

Via \acrshort{uart} wurden 2000 Messwerte erfasst. Ein Ausschnitt davon ist in Code \ref{code:gpio_toggle_with_hal_log}
gezeigt.

\lstinputlisting[language={}, label={code:gpio_toggle_with_hal_log}, caption={\acrshort{gpio} Toggle mit \acrshort{hal}}]{sourcecode/gpio_toggle_with_hal_log.txt}

Arithmetischer Mittelwert und Standardabweichung sind in Formel \ref{eq:gpio_toggle_with_hal} aufgeführt.

\begin{equation}\label{eq:gpio_toggle_with_hal}
    \begin{split}
        \overline{ToF} &= 6'375'887.9~ps\\
        \sigma         &= 1'059.2~ps
    \end{split}
\end{equation}
\myequations{\acrshort{gpio} Toggle mit \acrshort{hal}}

Da die \acrshort{cpu} mit 8~MHz läuft, lässt sich daraus schliessen, dass ein Pin-Toggle mit \acrshort{hal} ca. 50
\acrshort{cpu}-Cycles benötigt. Dies erscheint plausibel.

Histogramm und Boxplot sind in Abbildung~\ref{fig:gpio_toggle_with_hal_histogram} bzw.
\ref{fig:gpio_toggle_with_hal_boxplot} dargestellt.

\begin{figure}[H]
    \centering
    \includesvg[width=0.8\textwidth]{graphics/gpio_toggle_with_hal_histogram.svg}
    \caption{\acrshort{gpio} Toggle mit \acrshort{hal} - Histogramm}\label{fig:gpio_toggle_with_hal_histogram}
\end{figure}

\begin{figure}[H]
    \centering
    \includesvg[width=0.8\textwidth]{graphics/gpio_toggle_with_hal_boxplot.svg}
    \caption{\acrshort{gpio} Toggle mit \acrshort{hal} - Boxplot}\label{fig:gpio_toggle_with_hal_boxplot}
\end{figure}

Um die Resultate des \acrshort{tdc} zu validieren, wurde dieselbe Messung auch mittels \acrfull{dso} durchgeführt. Die
Messungen sind in Abbildung~\ref{fig:gpio_toggle_with_hal_dso} und \ref{fig:gpio_toggle_with_hal_dso_zoom} dargestellt.

\begin{figure}[H]
    \centering
    \includegraphics[width=0.8\textwidth]{graphics/gpio_toggle_with_hal_dso.png}
    \caption{\acrshort{gpio} Toggle mit \acrshort{hal} - \acrshort{dso}}\label{fig:gpio_toggle_with_hal_dso}
\end{figure}

\begin{figure}[H]
    \centering
    \includegraphics[width=0.8\textwidth]{graphics/gpio_toggle_with_hal_dso_zoom.png}
    \caption{\acrshort{gpio} Toggle mit \acrshort{hal} - \acrshort{dso} (Zoom)}\label{fig:gpio_toggle_with_hal_dso_zoom}
\end{figure}

\subsubsection{GPIO Toggle ohne HAL}\label{sec:gpio_toggle_without_hal}

Als nächstes wird gemessen wie lange es für die \acrshort{cpu} der \acrshort{mcu} dauert mit direktem Register-Zugriff
(via Pointer; ohne \acrshort{hal}-Library) zwei \acrshort{gpio}-Pins zu schalten.

In Code~\ref{code:gpio_toggle_without_hal} ist die Firmware-Implementation dazu gezeigt.

\lstinputlisting[language={C}, label={code:gpio_toggle_without_hal}, caption={\acrshort{gpio} Toggle ohne \acrshort{hal}}]{sourcecode/gpio_toggle_without_hal.c}

Vom \acrshort{tdc} wird also die Zeit gemessen zwischen Zeile 15 und 16 in Code~\ref{code:gpio_toggle_without_hal}. Dazu
wird \lstinline|SW1| mit \lstinline|stop_ele| verbunden und \lstinline|J3| kurzgeschlossen
(siehe Abbildung~\ref{fig:tdc_ele_signal}).

Via \acrshort{uart} wurden 2000 Messwerte erfasst. Ein Ausschnitt davon ist in Code
\ref{code:gpio_toggle_without_hal_log} gezeigt.

\lstinputlisting[language={}, label={code:gpio_toggle_without_hal_log}, caption={\acrshort{gpio} Toggle ohne \acrshort{hal}}]{sourcecode/gpio_toggle_without_hal_log.txt}

Arithmetischer Mittelwert und Standardabweichung sind in Formel \ref{eq:gpio_toggle_without_hal} aufgeführt.

\begin{equation}\label{eq:gpio_toggle_without_hal}
    \begin{split}
        \overline{ToF} &= 1'377'772.6~ps\\
        \sigma         &= 401.6~ps
    \end{split}
\end{equation}
\myequations{\acrshort{gpio} Toggle ohne \acrshort{hal}}

Da die \acrshort{cpu} mit 8~MHz läuft, lässt sich daraus schliessen, dass ein Pin-Toggle ohne \acrshort{hal} ca. 10
\acrshort{cpu}-Cycles benötigt. Dies erscheint plausibel.

Histogramm und Boxplot sind in Abbildung~\ref{fig:gpio_toggle_without_hal_histogram} bzw.
\ref{fig:gpio_toggle_without_hal_boxplot} dargestellt.

\begin{figure}[H]
    \centering
    \includesvg[width=0.8\textwidth]{graphics/gpio_toggle_without_hal_histogram.svg}
    \caption{\acrshort{gpio} Toggle ohne \acrshort{hal} - Histogramm}\label{fig:gpio_toggle_without_hal_histogram}
\end{figure}

\begin{figure}[H]
    \centering
    \includesvg[width=0.8\textwidth]{graphics/gpio_toggle_without_hal_boxplot.svg}
    \caption{\acrshort{gpio} Toggle ohne \acrshort{hal} - Boxplot}\label{fig:gpio_toggle_without_hal_boxplot}
\end{figure}

Um die Resultate des \acrshort{tdc} zu validieren, wurde dieselbe Messung auch mittels \acrfull{dso} durchgeführt. Die
Messungen sind in Abbildung~\ref{fig:gpio_toggle_without_hal_dso} und \ref{fig:gpio_toggle_without_hal_dso_zoom}
dargestellt.

\begin{figure}[H]
    \centering
    \includegraphics[width=0.8\textwidth]{graphics/gpio_toggle_without_hal_dso.png}
    \caption{\acrshort{gpio} Toggle ohne \acrshort{hal} - \acrshort{dso}}\label{fig:gpio_toggle_without_hal_dso}
\end{figure}

\begin{figure}[H]
    \centering
    \includegraphics[width=0.8\textwidth]{graphics/gpio_toggle_without_hal_dso_zoom.png}
    \caption{\acrshort{gpio} Toggle ohne \acrshort{hal} - \acrshort{dso} (Zoom)}\label{fig:gpio_toggle_without_hal_dso_zoom}
\end{figure}

\subsubsection{Unterschiedliche Kabellängen}\label{sec:different_cable_lengths}

Für diese Messung wird dasselbe Setup wie in Kapitel \ref{sec:gpio_toggle_without_hal} verwendet.

Anstelle eines Kurzschlusses von \lstinline|J3| (siehe Abbildung \ref{fig:tdc_ele_signal}) werden nun verschiedene
Kabellängen angeschlossen.

Es hat sich herausgestellt, dass eine kreisförmige Anordnung des Kabels wichtig ist. Denn bei einer Überlappung der
beiden Kabelenden werden kürze Zeiten gemessen. Dies hat mit der kapazitiven Kopplung zwischen den Leitern zu tun.

Die Resultate sind in Abbildung~\ref{fig:different_cable_lengths} dargestellt.

\begin{figure}[H]
    \centering
    \includesvg[width=0.9\textwidth]{graphics/different_cable_lengths.svg}
    \caption{Unterschiedliche Kabellängen}\label{fig:different_cable_lengths}
\end{figure}

Die arithmetischen Mittelwerte und Standardabweichungen sind in Tabelle \ref{tab:different_cable_lengths} aufgeführt.

\begin{table}[H]
    \mytable
        {|l|l|l|}
        {\textbf{Länge} & \textbf{Mittelwert} & \textbf{Standardabweichung}}
        {\length & \mean & \stddev}
        {tables/different_cable_lengths.csv}
    \caption{Unterschiedliche Kabellängen}\label{tab:different_cable_lengths}
\end{table}

Die Signal-Ausbreitungsgeschwindigkeit in Kupfer beträgt ca. 2/3 der Lichtgeschwindigkeit
\cite{firewallcx2025propagationdelay}. Um die Resultate in Tabelle \ref{tab:different_cable_lengths} zu validieren,
rechnen wir wie in Formel \ref{eq:cable_length} gezeigt, auf die Kabellänge zurück. Die Laufzeit bei 0~m wird dabei
abgezogen, um die Verzögerung zu kompensieren, welche durch das Schalten der \acrshort{gpio}s entsteht.

\begin{equation}\label{eq:cable_length}
    \begin{split}
        c_{cu} &\approx \frac{2}{3} \cdot c_0 = \frac{2}{3} \cdot 299'792'458~\frac{m}{s} \approx 200'000'000~\frac{m}{s}\\
        ToF_{n} &= ToF_{n_{abs}} - ToF_{0}\\
        l_{n}   &= ToF_{n} \cdot c_{cu}
    \end{split}
\end{equation}
\myequations{Zurückrechnen auf Kabellänge}

Die Resultate sind in Tabelle \ref{tab:different_cable_lengths_calc} dargestellt.

\begin{table}[H]
    \mytable
        {|l|l|l|}
        {\textbf{Tatsächliche Länge} & \textbf{ToF\_n} & \textbf{Zurückgerechnete Länge}}
        {\reallength & \tofn & \calclength}
        {tables/different_cable_lengths_calc.csv}
    \caption{Kabellängen zurückgerechnet}\label{tab:different_cable_lengths_calc}
\end{table}

Es fällt auf, dass die Resultate nicht genau übereinstimmen. Dies hat mehrere Ursachen: Die Ausbreitungsgeschwindigkeit
ist nicht genau bekannt und die tatsächlichen Kabellängen wurden nicht genau gemessen.

Es ist jedoch eine klare Korrelation zu erkennen.

\pagebreak

\subsection{Optische Messungen}

\pagebreak

\section{Fazit}

\pagebreak

\section{Anhang}

\subsection{TDC Treiber}\label{sec:tdc_driver}

In Code~\ref{code:tdc_driver_header} und \ref{code:tdc_driver_source} ist der selbst entwickelte Firmware-Treiber für
den \acrshort{tdc} dargestellt.

\lstinputlisting[language={c}, label={code:tdc_driver_header}, caption={TDC Driver (Header)}]{../firmware/Core/TDC/TDC.h}
\lstinputlisting[language={c}, label={code:tdc_driver_source}, caption={TDC Driver (Source)}]{../firmware/Core/TDC/TDC.c}

\subsection{Python Analyse}\label{sec:python_analyze}

In Code \ref{code:python_analyze} ist das Python-Skript zur Berechnung des arithmetischen Mittelwerts und der
Standardabweichung sowie zum Plotten des Histogramms und des Boxplots dargestellt.

\lstinputlisting[language={python}, label={code:python_analyze}, caption={Python Analyse}]{../utilities/analyze-tof.py}

In Code \ref{code:python_analyze_multi} ist das Python-Skript zur Berechnung des arithmetischen Mittelwerts und der
Standardabweichung sowie zum Plotten der Werte für mehrere Messungen dargestellt.

\lstinputlisting[language={python}, label={code:python_analyze_multi}, caption={Python Analyse (Multi)}]{../utilities/analyze-tofs.py}

%%%%%%%%%%%%%%%%%%%%%% CONTENT END %%%%%%%%%%%%%%%%%%%%%%

\pagebreak

\section*{Quellenverzeichnis}
\begin{flushleft}
\begingroup
\vspace{-25pt}
\renewcommand\refname{}
\renewcommand{\addcontentsline}[3]{} % Do not add bibliography to table of contents, as there is a separate subsection "Quellenverzeichnis"
\bibliography{references}
\endgroup
\end{flushleft}


\end{document}
