\documentclass[11pt,a4paper,hidelinks]{article}

\setlength{\headheight}{26pt}

% Packages
\usepackage[left=25mm, right=25mm, top=35mm, bottom=25mm]{geometry} % defines whitespaces on the edges
\usepackage[bottom, hang]{footmisc}
\usepackage{hyperref}             % for \nameref and cite
\usepackage{footnotebackref}      % used for getting back from the footnote to the main text
\usepackage{enumitem}
\usepackage{graphicx}
\usepackage{fancyhdr}
\usepackage{lastpage}
\usepackage[export]{adjustbox}
\usepackage{float}
\usepackage[ngerman]{babel}
\usepackage[ngerman]{datetime}
\usepackage{apacite}
\usepackage[utf8]{inputenc}
\usepackage{listings}
\usepackage{xcolor}
\usepackage{pdfpages}
\usepackage{url}
\usepackage{outlines}
\usepackage{tocloft}
\usepackage[utf8]{inputenc}
\usepackage{amsmath}
\usepackage{mathtools}
\usepackage[acronym]{glossaries}
\usepackage{titlesec}             % for \titleformat
\usepackage{textcomp}             % for \degree sign
\usepackage{gensymb}              % for \degree sign
\usepackage{csvsimple}            % for using csv-files for tables
\usepackage{lscape}               % for landscape mode
\usepackage{parskip}              % inserts space after paragraph automatically
\usepackage{svg}                  % used to include svg with the help of inkscape which converts svg to pdf
\usepackage{wasysym}              % for \diameter symbol
\usepackage{xfrac}                % for \sfrac

\frenchspacing  % Use same space size between words and between sentences

% change section titles' font size
\titleformat*{\section}{\huge\bfseries}
\titleformat*{\subsection}{\LARGE\bfseries}
\titleformat*{\subsubsection}{\Large\bfseries}
\titleformat{\paragraph}[hang]{\large\bfseries}{\theparagraph}{1em}{}
\titleformat{\subparagraph}[hang]{\normalsize\bfseries}{\thesubparagraph}{1em}{}

\definecolor{codegreen}{rgb}{0,0.6,0}
\definecolor{codegray}{rgb}{0.5,0.5,0.5}
\definecolor{codepurple}{rgb}{0.58,0,0.82}
\definecolor{codeorange}{rgb}{1,0.5,0.15}
\definecolor{backcolour}{rgb}{0.9,0.9,0.9}

\lstdefinestyle{mystyle}{
    backgroundcolor=\color{backcolour},
    commentstyle=\color{codegreen},
    keywordstyle=\color{magenta},
    numberstyle=\tiny\color{codegray},
    stringstyle=\color{codepurple},
    basicstyle=\ttfamily\footnotesize,
    breakatwhitespace=false,
    breaklines=true,
    captionpos=b,
    keepspaces=true,
    numbers=left,
    numbersep=5pt,
    showspaces=false,
    showstringspaces=false,
    showtabs=false,
    tabsize=4
}
\lstset{style=mystyle}

\lstset{
    literate={~} {$\sim$}{1}
}

\lstset{%
    breaklines=true,
    breakatwhitespace=true,
}

\DeclarePairedDelimiter\abs{\lvert}{\rvert} % make scalable absolute stripes
\DeclarePairedDelimiter\parenth{(}{)} % make scalable parentheses

\let\phi\varphi{} % change style of \phi sign

\setlength{\parindent}{0mm} % disable paragraph indent

\newdateformat{mydate}{\THEDAY{. }\monthnamengerman[\THEMONTH] \THEYEAR}

\renewcommand{\listfigurename}{}
\renewcommand\contentsname{Inhaltsverzeichnis}
\renewcommand\lstlistingname{Code}

\makeglossaries{}

% Header/Footer Setting
\setlength\footnotemargin{15pt}
\pagestyle{fancy}
\fancyhf{}
\renewcommand{\footrulewidth}{0.4pt} % footer line
\rhead{\textbf{\vUniversity}\\\vModule}
\lhead{\textbf{\vTitle}\\
    Projektarbeit}
\lfoot{\vAuthorFirstName{} \vAuthorLastName}
\cfoot{\mydate\today}
\rfoot{S.~\thepage~/~\pageref{LastPage}}

% Redefine the plain page style, otherwise there is no header and footer for chapter pages
\fancypagestyle{plain}{%
    \fancyhf{}
    \renewcommand{\footrulewidth}{0.4pt} % footer line
    \rhead{\textbf{\vUniversity}\\\vModule}
    \lhead{\textbf{\vTitle}\\
        Projektarbeit}
    \lfoot{\vAuthorFirstName{} \vAuthorLastName}
    \cfoot{\mydate\today}
    \rfoot{S.~\thepage~/~\pageref{LastPage}}
}

\bibliographystyle{apacite}

% Settings for the equation list
\newcommand{\listequationsname}{}
\newlistof{myequations}{equ}{\listequationsname}
\renewcommand{\cftmyequationsaftersnum}{\hfill}
\renewcommand{\cftmyequationspresnum}{\hfill}
\setlength{\cftmyequationsnumwidth}{3.5em}
\newcommand{\myequations}[1]{%
\addcontentsline{equ}{myequations}{\protect\numberline{\theequation}#1}\par}

\newcommand{\mytable}[4]
{
    \centering
    \begin{tabular}{#1}\hline%
        #2 \\ \hline
        \csvreader[
            separator=semicolon,
            head to column names,
            late after line=\\,
        ]{#4}{}{#3}
        \hline
    \end{tabular}
}


% Variables
\newcommand{\vTitle}{DIY Optische ToF Distanzmessung}
\newcommand{\vModule}{CAS Sensorik und Sensor Signal Conditioning}
\newcommand{\vAuthorFirstName}{Matthias Schär,}
\newcommand{\vAuthorLastName}{Timon Burkard}
\newcommand{\vUniversity}{OST -- Ostschweizer Fachhochschule}
\newcommand{\vDegree}{CAS Sensorik und Sensor Signal Conditioning}
\newcommand{\vSemester}{HS24}
\newcommand{\vProfessor}{Prof. Guido Keel}
\newcommand{\vCity}{Rapperswil}
\newcommand{\vAbstract}{Die vorliegende Projektarbeit befasst sich mit der Entwicklung eines\dots}

% Acronyms
\newacronym{pcb}{PCB}{Printed Circuit Board}
\newacronym{tof}{ToF}{Time of Flight}
\newacronym{diy}{DIY}{Do It Yourself}
\newacronym{tdc}{TDC}{Time to Digital Converter}

\begin{document}

\title{\begin{huge}\textbf{\vTitle}\end{huge}\\
\ \\ \vModule}
\author{\\\\\textbf{\vAuthorFirstName{} \vAuthorLastName}\\
    \vUniversity{}\\
    \\\\}

\date{\mydate\today}
\maketitle\thispagestyle{empty}  % removes footer from first page

\vspace{70pt}

\begin{figure}[H]
    \centering
    \begin{minipage}[b]{0.35\textwidth}
        \includesvg[width=\textwidth]{graphics/OST_Logo.svg}\label{fig:OST_Logo}
    \end{minipage}
\end{figure}

\pagebreak

\thispagestyle{empty}

\textbf{}
\vspace{5mm}

\begin{flushleft}
    \textbf{\large{\vModule{} an der \vUniversity{}}}
\end{flushleft}

\begin{flushleft}
    \begin{small}
        \begin{tabular}{@{}lll}
            \\
            \textbf{Titel}                 & & \textbf{\vTitle}\\
            \\
            \textbf{Diplomandin/Diplomand} & & \textbf{\vAuthorFirstName{} \vAuthorLastName}\\
            \\
            \textbf{Studiengang}           & & \textbf{\vDegree}\\
            \\
            \textbf{Semester}              & & \textbf{\vSemester}\\
            \\
            \textbf{Dozentin/Dozent}       & & \textbf{\vProfessor}\\
        \end{tabular}
    \end{small}
\end{flushleft}

\vspace{10mm}

% abstracts
\begin{small}
    \textbf{Abstract}

    \vAbstract{}
    \vspace{15mm}
\end{small}

% copyright
\begin{flushleft}
    \begin{small}
        \vCity, \date{\mydate\today} \\
        \textbf{\textcopyright\hspace{1mm}\vAuthorFirstName{} \vAuthorLastName, \vUniversity{}}
    \end{small}
\end{flushleft}

\mbox{}
\vfill

\pagebreak

\section*{Inhaltsverzeichnis}
\vspace{-25pt}
\renewcommand*\contentsname{}
\tableofcontents

\pagebreak

\section*{Abkürzungsverzeichnis}
\vspace{-25pt}
\printglossary[type=\acronymtype,title={}]

\pagebreak

\section*{Abbildungsverzeichnis}
\vspace{-25pt}
\renewcommand{\listfigurename}{}
\listoffigures

\pagebreak

\section*{Formelverzeichnis}
\vspace{-25pt}
\listofmyequations{}

\pagebreak

\section*{Tabellenverzeichnis}
\vspace{-25pt}
\renewcommand{\listtablename}{}
\listoftables

\pagebreak

\section*{Codeverzeichnis}
\vspace{-25pt}
\renewcommand{\lstlistlistingname}{}
\lstlistoflistings{}

\pagebreak


%%%%%%%%%%%%%%%%%%%%%% CONTENT START %%%%%%%%%%%%%%%%%%%%%%

\section{Einleitung}

\section{Theorie}

\subsection{Time of Flight}
\subsection{Photostrom}
\subsection{Transimpedanzverstärker}

\section{Umsetzung}

\subsection{Firmware}
\subsection{Schaltungen}

\subsubsection{Selective Input Voltage}
\subsubsection{Nucleo Board}
\subsubsection{TDC Electrical Signal}
\subsubsection{TDC Optical Signal}
\subsubsection{Oscillator For TDCs}
\subsubsection{Powersupply Separation}
\subsubsection{Laser Driver}
\subsubsection{Photo Receiver}
\subsubsection{Decoupling Capacitors}

\subsection{Layout}
\subsection{3D View}
\subsection{Komponenten}

\section{Simulationen}

\subsection{Laser Treiber}
\subsection{Transimpedanzverstärker}

\section{Messungen}

\section{Fazit}

\section{Anhang}

%%%%%%%%%%%%%%%%%%%%%% CONTENT END %%%%%%%%%%%%%%%%%%%%%%

\pagebreak

\section*{Quellenverzeichnis}
\begin{flushleft}
\begingroup
\vspace{-25pt}
\renewcommand\refname{}
\renewcommand{\addcontentsline}[3]{} % Do not add bibliography to table of contents, as there is a separate subsection "Quellenverzeichnis"
\bibliography{references}
\endgroup
\end{flushleft}


\end{document}
