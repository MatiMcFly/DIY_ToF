In diesem Kapitel werden abschliessende Gedanken gesammelt. Es soll dazu dienen, einige Punkte aufzuzählen, die gut gelaufen
sind, aber auch jene bei denen Verbesserungspotential identifiziert wurde.

\subsection{Ausblick}
Grundsätzlich kann bestätigt werden, dass sich ein \acrshort{tof}-System, so wie es sich die Studierenden vorstellen,
entwickeln und betreiben lässt. Ein nächster Schritt könnte beispielsweise sein, dass die Optik besser aufeinander
abgestimmt ist. Dies würde es zum einen erlauben, über weitere Distanzen hinweg Messungen machen zu können. Des weiteren
sollte es ebenfalls möglich sein, auch Targets mit nicht oder schwach reflektierenden Oberflächenbeschaffenheiten zu detektieren.

Auch allgemein bei der Ansteuerung der \acrshort{tdc}s konnte gibt es Verbesserungspotential. So könnte beispielsweise
mit einer schneller getakteten \acrshort{mcu} der Jitter bei \lstinline|START|- und \lstinline|STOP|-Signal minimiert
werden, was zu Messresultaten mit noch weniger Streuung führen wird.

Abschliessend kann gesagt werden, dass es durchaus möglich ist, mit einem TDC7200 eine \acrshort{tof}-Messung zu machen,
bei welcher die zeitliche Auflösung des Mess-ICs selber die limitierende Grösse ist.

\subsection{Lessons Learned}
Wie die Messresultate und auch der Ausblick bereits vermuten lassen, hätte primär beim Auslegen der Optoelektronik etwas
mehr Zeit investiert werden können. Die initialen, überschlagsmässigen Schätzungen und Rechnungen bezüglich der gewünschten
Limitierungen des Systems hätten kritischer hinterfragt werden sollen. Initial war es das Ziel der Studenten, Messungen
auf Distanzen von bis zu 10~m mit einer Auflösung von etwa 10~cm zu machen. Herausgestellt hat sich am Ende jedoch, dass
nicht die zeitliche Auflösung, sondern vielmehr die Distanz, resp. die Stärke des Messsignals problematisch ist.

Mit der Fabrikation des \acrshort{tof}-Demonstrators ist das Studenten-Team relativ zufrieden. Beim erneuten Einsatz so
kleiner SMD-Komponenten, wie beispielsweise beim \acrshort{tia} lohnt es sich jedoch darüber zu sinnieren, ob nicht eine
maschinelle Bestückung der händischen Bestückung vorzuziehen ist. Durch die manuelle Bestückung wurde wohl keine Zeit
eingespart, da diese auch einige Studen über mehrere Abende verteilt in Anspruch nahm.

\subsection{Danksagung}
Wir möchten uns an dieser Stelle herzlich bei den beiden Dozenten Prof. Guido Keel und Michael Lehmann bedanken. Während
der Laufzeit des Projekts sind sie uns jederzeit mit Rat und Tat zur Seite gestanden und haben das Projekt mit grossem
Interesse verfolgt. Auch für das optische Leihmaterial bedanken wir uns. Dieses hat es wohl ermöglicht, dass wir bei der
optischen Messung einen Teilerfolg erzielen durften.

Weiter bedanken wir uns bei Robin Burkard, welcher unser Schema reviewt hat. Dank seinen wertvollen Tipps konnte das
\acrshort{pcb} so störfrei wie möglich in Betrieb genommen werden.
