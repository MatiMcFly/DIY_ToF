In diesem Kapitel werden abschliessende Gedanken gesammelt. Es soll dazu dienen einen Ausblick zu geben für mögliche,
weiterführende Arbeiten. Es sollen auch einige Punkte aufzuzählen, die gut gelaufen sind, aber auch jene bei denen
Verbesserungspotential identifiziert wurde.

\subsection{Ausblick}\label{sec:ausblick}

Grundsätzlich kann bestätigt werden, dass sich ein \acrshort{tof}-System, so wie es sich die Studierenden vorstellen,
entwickeln und betreiben lässt. Nachfolgend werden ein paar mögliche Verbesserungen und weiterführende Messungen
aufgelistet.

\begin{enumerate}
    \item Die Optik besser aufeinander abstimmen. Dies würde es zum einen erlauben, über weitere Distanzen hinweg
          Messungen machen zu können. Des weiteren sollte es ebenfalls möglich sein, auch Targets mit nicht oder schwach
          reflektierenden Oberflächen-Beschaffenheiten zu detektieren.
    \item Für die Messungen mit dem Spiegel würde es Sinn machen mit einer Linse mit grösserer Brennweite (oder mit
          derselben Linse mit etwas weniger Distanz zum \acrshort{pcb}) auszuprobieren. Dies hätte den Vorteil, dass der
          Spiegel nicht ganz so genau ausgerichtet werden muss. Der Nachteil wäre etwas weniger Photostrom bei parallel
          einfallendem Licht.
    \item Bei der Ansteuerung der \acrshort{tdc}s gibt es Verbesserungspotential. So könnte beispielsweise mit einer
          schneller getakteten \acrshort{mcu} der Jitter bei \lstinline|START|- und \lstinline|STOP|-Signal minimiert
          werden, was zu Messresultaten mit noch weniger Streuung führen wird.
    \item Eine andere Möglichkeit um die Streuung zu senken, wäre es eine externe Clock-Quelle zu verwenden.
    \item Eventuell ist es möglich auf die künstlich eingefügte Verzögerungszeit von einem Taktzyklus zu verzichten.
          Dies wäre der Fall, falls im Messpfad bereits genug Verzögerung entsteht, um die minimale Messdauer von 12~ns
          zu überschreiten. Falls nicht, könnte die künstliche Verzögerung anstatt von der \acrshort{mcu} auch mittels
          separater Schaltung (mit möglichst tiefem Jitter) umgesetzt werden.
    \item Beim verwendeten Nucleo gibt es die Möglichkeit, die Anstiegszeit der \acrshort{gpio} zu konfigurieren. Mit
          schnellerer Anstiegszeit lassen sich evtl. die Messresultate verbessern.
    \item Optischer Empfangspfad: Es sollten Messungen mit unterschiedlichen Verstärkungen des \acrshort{tia}
          durchgeführt werden, um die Distanz mit Spiegel zu erhöhen und Messungen gegen die Wand oder andere Targets zu
          ermöglichen. Damit zusammenhängend sollte die Schaltschwelle des Komparators genauer ausgetestet und
          feinjustiert werden.
    \item Optischer Sendepfad: Es lohnt sich mit höherer Sendeleistung zu experimentieren. Aufgrund der 5~V Speisung
          der Laser-Diode sind wir mit dem $5~\Omega$ Vorwiderstand schon fast am Limit. In einer nächsten
          \acrshort{pcb}-Version würde es sich lohnen mit einem Schalter die Möglichkeit zur 12~V Speisung vorzusehen,
          analog dazu wie dies für die Photodiode bereits umgesetzt ist.
    \item Optische Messungen: Es lohnt sich zu analysieren, weshalb die Messresultate in zwei Clustern (ca. 2~ns
          auseinander liegend) verteilt sind.
    \item Für weitere Messungen mit einem Spiegel lohnt es sich einen präzisen, mechanischen Versuchsaufbau zu
          konstruieren. Bei diesem könnte der Spiegel besser fixiert und eingestellt werden. Von Hand ist es schwierig
          reproduzierbare Ergebnisse zu erzielen.
\end{enumerate}

Abschliessend kann gesagt werden, dass es durchaus möglich ist, mit einem TDC7200 eine \acrshort{tof}-Messung zu machen,
bei welcher die zeitliche Auflösung des Mess-\acrshort{ic}s selber die limitierende Grösse ist.

\subsection{Lessons Learned}

Wie die Messresultate und auch der Ausblick bereits vermuten lassen, hätte primär beim Auslegen der Optoelektronik etwas
mehr Zeit investiert werden können. Die initialen, überschlags-mässigen Schätzungen und Rechnungen bezüglich der gewünschten
Limitierungen des Systems hätten kritischer hinterfragt werden sollen. Initial war es das Ziel der Studenten, Messungen
auf Distanzen von bis zu 10~m mit einer Auflösung von etwa 10~cm zu machen. Herausgestellt hat sich am Ende jedoch, dass
nicht die zeitliche Auflösung, sondern vielmehr die Distanz, resp. die Stärke des Messsignals problematisch ist.

Mit der Fabrikation des \acrshort{tof}-Demonstrators ist das Studenten-Team zufrieden. Beim erneuten Einsatz so kleiner
SMD-Komponenten, wie beispielsweise beim \acrshort{tia} lohnt es sich jedoch darüber zu sinnieren, ob nicht eine
maschinelle Bestückung der händischen Bestückung vorzuziehen ist. Durch die manuelle Bestückung wurde wohl keine Zeit
eingespart, da diese auch einige Stunden über mehrere Abende verteilt in Anspruch nahm.

\subsection{Danksagung}

Wir möchten uns an dieser Stelle herzlich bei den beiden Dozenten Prof. Guido Keel und Michael Lehmann bedanken. Während
der Laufzeit des Projekts sind sie uns jederzeit mit Rat und Tat zur Seite gestanden und haben das Projekt mit grossem
Interesse verfolgt. Auch für das optische Leihmaterial bedanken wir uns. Dieses hat uns ermöglicht, bei der optischen
Messung einen Teilerfolg erzielen zu können.

Weiter bedanken wir uns bei Robin Burkard, der für uns ein Layout-Review durchgeführt hat. Dank seinen wertvollen Tipps
konnte das \acrshort{pcb} so störfrei wie möglich in Betrieb genommen werden.
