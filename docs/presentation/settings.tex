\usetheme{Frankfurt}
\usecolortheme{default}

\usepackage[ngerman]{babel}
\usepackage[square,numbers,super]{natbib}
\usepackage{listings}
\usepackage{textpos}
\usepackage{tabto}
\usepackage{caption}
\usepackage{fbox}                     % to add border to figures, if needed
\usepackage{tikz}                     % to place red rounded boxes to highlight spots
\usepackage{xcolor}
\usepackage{xpatch}                   % to add grey background to \lstinline
\usepackage{realboxes}                % to add grey background to \lstinline
\usepackage{calc}                     % used for \rightHeaderWidth
\usepackage{varwidth}                 % used for \rightHeaderWidth
\usepackage{lmodern}                  % used for \setlistspacing
\usepackage[export]{adjustbox}        % used to left/right align graphics in a figure
\usepackage[normalem]{ulem}           % used to strike through text: \sout{}
\usepackage[inkscapelatex=false]{svg} % used to include svg with the help of inkscape which converts svg to pdf (inkscapelatex=false: prevent latex from changing the font within the svg graphic)
\usepackage[loop,autoplay]{animate}   % used to insert GIF animations with the help op ImageMagick which converts gif to multiple png
\usepackage{mathtools}                % used for DeclarePairedDelimiter
\usepackage{amsmath}                  % needed for amssymb
\usepackage{amssymb}                  % used for \circledast
\usepackage{trfsigns}                 % transformation signs for fourier and laplace
\usepackage{csvsimple}                % for using csv-files for tables

\makeatletter
\newcommand{\setlistspacing}[2]{\def\@ld{#1}\expandafter\def\csname
@list\romannumeral\@ld\endcsname{\leftmargin\csname
leftmargin\romannumeral\@ld\endcsname
              \topsep#2
              \parsep0\p@ \@plus\p@
              \itemsep#2}}
\makeatother
\setlistspacing{1}{4ex}

\captionsetup{labelformat=empty,labelsep=none,font=scriptsize} % don't show caption label only caption text

\renewcommand{\bibsection}{} % don't create a section for bibliography

\urlstyle{same} % use text font for URLs
\renewcommand\UrlFont{\color{blue}} % make URLs blue

\definecolor{darkblue}{rgb}{0.05, 0.13, 0.28}
\definecolor{grey}{rgb}{0.4, 0.5, 0.6}
\definecolor{black}{rgb}{0, 0, 0}

\setbeamercolor{palette primary}{bg=darkblue,fg=white}
\setbeamercolor{palette secondary}{bg=darkblue,fg=white}
\setbeamercolor{palette tertiary}{bg=darkblue,fg=white}
\setbeamercolor{palette quaternary}{bg=black,fg=white}
\setbeamercolor{structure}{fg=darkblue} % itemize, enumerate, etc
\setbeamercolor{section in toc}{fg=darkblue}
\setbeamercolor{caption name}{fg=black}

\setbeamercolor{subsection}{bg=grey,fg=white}
\setbeamercolor{subsection in head/foot}{bg=grey,fg=white}

\setbeamertemplate{navigation symbols}{} % hides navigation buttons at bottom

\makeatletter
\setbeamertemplate{footline}
{
    \leavevmode%
    \hbox{%
    \begin{beamercolorbox}[wd=.5\paperwidth,ht=2.25ex,dp=1ex,right]{author in head/foot}%
        \usebeamerfont{author in head/foot}\insertauthor\hspace*{2ex}
    \end{beamercolorbox}%
    \begin{beamercolorbox}[wd=.4\paperwidth,ht=2.25ex,dp=1ex,left]{subsection in head/foot}%
        \hspace*{2ex}\usebeamerfont{title in head/foot}\inserttitle{}
    \end{beamercolorbox}%
    \begin{beamercolorbox}[wd=.1\paperwidth,ht=2.25ex,dp=1ex,right]{subsection in head/foot}%
        \usebeamerfont{title in head/foot}\insertframenumber{} / \inserttotalframenumber\hspace*{2ex}
    \end{beamercolorbox}}%
    \vskip0pt%
}
\makeatother

\setbeamerfont{date}{size=\scriptsize}
\setbeamerfont{institute}{size=\normalfont}
\setbeamerfont{subsection}{size=\large}

\hypersetup{
    colorlinks=true,
    urlcolor=blue,
    linkcolor=.,
    anchorcolor=.,
    citecolor=.,
    filecolor=.
}

\defbeamertemplate*{title page}{customized}[1][]
{
    \begin{beamercolorbox}[sep=8pt,center,shadow=true,rounded=true]{title}
        \usebeamerfont{title}\inserttitle\par%
    \end{beamercolorbox}

    \centering

    \vspace{30pt}

    \usebeamerfont{institute}\insertinstitute\par

    \vspace{10pt}

    \footnotesize{\vModule}

    \vspace{30pt}

    \usebeamerfont{author}\insertauthor\par

    \vspace{40pt}

    \usebeamerfont{date}\insertdate\par
}

\newcommand{\rightHeaderWidth}[0]
{
    \widthof{
        \fbox{
            \begin{varwidth}{\textwidth}
                \tiny{\insertshortinstitute{}}
                \tiny{\vModule}
            \end{varwidth}
        }
    }
}

\addtobeamertemplate{frametitle}{}{%
    \begin{textblock*}{100mm} (153mm-\rightHeaderWidth{},-7.2mm)
        \tiny{\insertshortinstitute{}}
        \vspace{2pt}
        \tiny{\vModule}
    \end{textblock*}}

\AtBeginSection[]{
    \begin{frame}
    \vfill
    \centering
    \begin{beamercolorbox}[sep=8pt,center,shadow=true,rounded=true]{title}
        \usebeamerfont{title}\insertsectionhead\par%
    \end{beamercolorbox}
    \vfill
    \end{frame}
}

\AtBeginSubsection[]{
    \begin{frame}
    \vfill
    \centering
    \begin{beamercolorbox}[sep=8pt,center,shadow=true,rounded=true]{title}
        \usebeamerfont{title}\insertsectionhead\par%
    \end{beamercolorbox}
    \vspace{0.5cm}
    \begin{beamercolorbox}[sep=8pt,center,shadow=true,rounded=true]{subsection}
        \usebeamerfont{subsection}\insertsubsectionhead\par%
    \end{beamercolorbox}
    \vfill
    \end{frame}
}

\setbeamertemplate{frametitle continuation}{} % deactivate roman numbering for frame with [allowframebreaks]

\setbeamercolor{block title}{bg=darkblue,fg=white}

\newcommand{\question}[1]
{
    \begin{block}{Frage}
        #1
    \end{block}
}

\newcommand{\attention}[1]
{
    \begin{alertblock}{Wichtig}
        #1
    \end{alertblock}
}

\newcommand{\exercise}[2][n]
{
    \begin{exampleblock}{Übung}
        #2
    \end{exampleblock}

    \ifx #1n
    \else
        \begin{itemize}
            \item<2->[$\Rightarrow$] Lösungen werden via GitLab zur Verfügung gestellt:
            \url{#1}
        \end{itemize}
    \fi
}

\renewcommand{\quote}[3]
{
    \begin{frame}{#1}
        \begin{block}{}
            {\emph{\dq #2\dq}}
            \vskip5mm
            \hspace*\fill{\small-- #3}
        \end{block}
    \end{frame}
}

\newcommand{\goals}[1]
{
    {
        \setlistspacing{1}{1ex}

        \begin{frame}{Ziele}
            Die Studierenden

            \begin{itemize}
                #1
            \end{itemize}

            \vspace{10pt}

            \begin{figure}
                \includegraphics[width=0.25\textwidth]{includes/goal.pdf}
            \end{figure}
        \end{frame}
    }
}

\newcommand{\questions}[0]
{
    \begin{frame}{Fragen?}
        \begin{figure}
            \includegraphics[width=0.4\textwidth]{includes/questions.jpg}
        \end{figure}
    \end{frame}
}

\definecolor{codegreen}{rgb}{0,0.6,0}
\definecolor{codegray}{rgb}{0.5,0.5,0.5}
\definecolor{codepurple}{rgb}{0.58,0,0.82}
\definecolor{codeorange}{rgb}{1,0.5,0.15}
\definecolor{backcolour}{rgb}{0.9,0.9,0.9}

\colorlet{codegreen_transparent}{codegreen!30}
\colorlet{codegray_transparent}{codegray!30}
\colorlet{codepurple_transparent}{codepurple!30}
\colorlet{codeorange_transparent}{codeorange!30}
\colorlet{backcolour_transparent}{backcolour!30}
\colorlet{magenta_transparent}{magenta!30}
\colorlet{red_transparent}{red!30}
\colorlet{black_transparent}{black!30}

\lstdefinestyle{mystyle}{
    language=C,
    backgroundcolor=\color{backcolour},
    commentstyle=\lst@ifdisplaystyle\color{codegreen}\fi, % don't apply for \lstinline
    keywordstyle=\lst@ifdisplaystyle\color{magenta}\fi,   % don't apply for \lstinline
    stringstyle=\lst@ifdisplaystyle\color{codepurple}\fi, % don't apply for \lstinline
    numberstyle=\tiny\color{codegray},
    basicstyle=\ttfamily\scriptsize,
    breakatwhitespace=false,
    breaklines=true,
    captionpos=b,
    keepspaces=true,
    numbers=left,
    numbersep=5pt,
    showspaces=false,
    showstringspaces=false,
    showtabs=false,
    tabsize=4,
    morekeywords={int8_t, uint8_t, int16_t, uint16_t, int32_t, uint32_t, int64_t, uint64_t, bool, true, false},
    moredelim=[is][\color{red}]{§}{§}
}
\lstset{style=mystyle}

\lstset{
    literate={~} {$\sim$}{1},
    breaklines=true,
    breakatwhitespace=true,
    upquote=true, % straight single quotes
}

\lstdefinestyle{transparent}{
    language=C,
    backgroundcolor=\color{backcolour_transparent},
    commentstyle=\lst@ifdisplaystyle\color{codegreen_transparent}\fi, % don't apply for \lstinline
    keywordstyle=\lst@ifdisplaystyle\color{magenta_transparent}\fi,   % don't apply for \lstinline
    stringstyle=\lst@ifdisplaystyle\color{codepurple_transparent}\fi, % don't apply for \lstinline
    numberstyle=\tiny\color{codegray_transparent},
    basicstyle=\ttfamily\scriptsize\color{black_transparent}
    breakatwhitespace=false,
    breaklines=true,
    captionpos=b,
    keepspaces=true,
    numbers=left,
    numbersep=5pt,
    showspaces=false,
    showstringspaces=false,
    showtabs=false,
    tabsize=4,
    morekeywords={int8_t, uint8_t, int16_t, uint16_t, int32_t, uint32_t, int64_t, uint64_t, bool, true, false},
    moredelim=[is][\color{red_transparent}]{§}{§}
}

\newcommand{\sidecaption}[1]
{
    \begin{minipage}{0.05\textwidth}%
        \caption{#1}
    \end{minipage}
}

% \highlight[color]{x1,y1}{x2,y2}
% Places a rounded rectangle at an absolute position on the page.
% Arguments:
%   #1 (optional) - Color of the rectangle (defaults to red)
%   #2            - Coordinates (x,y) of first rectangle corner
%   #3            - Coordinates (x,y) of second rectangle corner
\newcommand{\highlight}[3][red]
{
    \begin{tikzpicture}[overlay]
        \draw[#1, rounded corners, thick] (#2) rectangle (#3) {};
    \end{tikzpicture}
}

% \textbox[color]{x,y}{text}
% Places a text box at an absolute position on the page.
% Arguments:
%   #1 (optional) - Color of the text (defaults to red)
%   #2            - Coordinates (x,y) where the text box should be placed
%   #3            - The text content to display
\newcommand{\textbox}[3][red]
{
    \begin{picture}(0,0)
        \put(#2){\makebox(0,0)[lt]{\textcolor{#1}{#3}}}
    \end{picture}
}

\newcommand{\codehighlight}[5][\ttfamily\scriptsize]
{
    \ifnum#3>1
        \lstinputlisting[firstnumber=1, firstline=1, lastline=\number\numexpr#3-1\relax, language=#2, style=transparent, basicstyle=#1\color{black_transparent}, showlines=true]{#5}
        \vspace{-12pt}
    \fi
    \lstinputlisting[firstnumber=#3, firstline=#3, lastline=#4, language=#2, style=mystyle, basicstyle=#1, showlines=true]{#5}
    \vspace{-12pt}
    \lstinputlisting[firstnumber=\number\numexpr#4+1\relax, firstline=\number\numexpr#4+1\relax, lastline=100, language=#2, style=transparent, basicstyle=#1\color{black_transparent}, showlines=true]{#5}
}

% to be able to remove the final frames (questions, references) from the final section
\makeatletter
\let\beamer@writeslidentry@miniframeson=\beamer@writeslidentry%
\def\beamer@writeslidentry@miniframesoff{%
  \expandafter\beamer@ifempty\expandafter{\beamer@framestartpage}{}% does not happen normally
  {%else
    % removed \addtocontents commands
    \clearpage\beamer@notesactions%
  }
}
\newcommand*{\miniframeson}{\let\beamer@writeslidentry=\beamer@writeslidentry@miniframeson}
\newcommand*{\miniframesoff}{\let\beamer@writeslidentry=\beamer@writeslidentry@miniframesoff}
\makeatother

\newcommand{\mytable}[4]
{
    \centering
    \begin{tabular}{#1}\hline%
        #2 \\ \hline
        \csvreader[
            separator=semicolon,
            head to column names,
            late after line=\\,
        ]{#4}{}{#3}
        \hline
    \end{tabular}
}

\newcommand{\iconelectrical}{
    \begin{tikzpicture}[remember picture, overlay]
        \node at ([xshift=0.8cm, yshift=-2.3cm]current page.north west) {
            \includesvg[width=1cm]{graphics/icon_electrical.svg}
        };
    \end{tikzpicture}
}

\newcommand{\iconoptical}{
    \begin{tikzpicture}[remember picture, overlay]
        \node at ([xshift=0.8cm, yshift=-2.3cm]current page.north west) {
            \includesvg[width=1cm]{graphics/icon_optical.svg}
        };
    \end{tikzpicture}
}

% usable in math mode
\DeclarePairedDelimiter\abs{\lvert}{\rvert}     % make scalable absolute stripes
\DeclarePairedDelimiter\parenth{(}{)}           % make scalable parentheses
\DeclarePairedDelimiter\brackets{[}{]}          % make scalable brackets
\DeclarePairedDelimiter\floor{\lfloor}{\rfloor} % make scalable floor
\DeclarePairedDelimiter\ceil{\lceil}{\rceil}    % make scalable ceil
